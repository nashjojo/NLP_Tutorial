%%%%%%%%%%%%%%%%%%%%% chapter.tex %%%%%%%%%%%%%%%%%%%%%%%%%%%%%%%%%
%
% sample chapter
%
% Use this file as a template for your own input.
%
%%%%%%%%%%%%%%%%%%%%%%%% Springer-Verlag %%%%%%%%%%%%%%%%%%%%%%%%%%
%\motto{Use the template \emph{chapter.tex} to style the various elements of your chapter content.}
\chapter{自然语言处理技术展望}
\label{prospect}

随着算法的持续改进以及数据和算力的持续增长,自然语言处理技术在最近十年间取得了长足的进步,基于深度神经网络的模型在多个自然语言处理任务中刷新了历史记录。这些新兴的自然语言处理技术在改善了传统自然语言处理任务的效果的同时,也极大地促进了一些相关领域的发展。例如,在信息检索领域,自然语言理解技术能够帮助搜索引擎更准确地识别用户的潜在需求;文本摘要技术能够帮助搜索引擎更好地刻画文本中的重要信息;自然语言对话技术有极大的潜力带来搜索引擎和用户交互方式的深刻变革。在语音识别领域,最新的语言模型技术(例如BERT、XLNet等\cite{devlin2018bert, yang2019xlnet, song2019topic})能够帮助语音识别系统更好地控制识别结果在语法和语义层面的合理性;通过自然语言理解技术和语音识别系统的无缝结合,智能音箱等设备正在逐步走进人们的日常生活,也重构了AIOT行业的生态版图。

然而,当前的自然语言处理技术还面临着诸多挑战:

\begin{itemize}
  \item 基于深度神经网络的自然语言处理模型往往是Model-blind的,其效果严重依赖大量的数据和超强的算力,这通常会导致极高的模型成本,也极大地限制了自然语言处理技术的公平性和普惠性。因此,以贝叶斯程序学习 (Bayesian Program Learning) \cite{lake2015human}为代表的Model-based小样本学习正在逐步引起研究人员的关注。通过引入领域相关的先验知识,贝叶斯程序学习等技术可以有效降低模型训练中对于大量训练数据的依赖,节省数据标注成本和算力开销。
  \item 当前的自然语言处理技术通常只能发现数据之间的关联(Association) 而不能发现数据之间的因果关系(Causality)。例如,当前的对话系统技术虽然可以生成一些似是而非的语句,但是从多轮的语境来看,这些机器生成的对话往往存在非常严重的逻辑问题。导致这种现象的重要原因是神经网络模型通常只是对训练数据做曲线拟合, 缺乏对数据背后的复杂关系的深刻理解。最近日益受到科研人员重视的因果推理 (Causal Inference) \cite{pearl2010introduction} 技术有助于人们对数据进行因果关系分析,该技术和传统自然语言处理技术的深度结合有潜力使自然语言处理在认知层面更上一层楼。
  \item 随着诸多类似欧盟General Data Protection Regulation (GDPR)法规的陆续出台,数据监管正在变得日益严格,单一研发机构获取大量的训练语料变得越发困难。如何在合规前提下统筹利用行业内的数据助力自然语言处理的研发任务是当前需要研究的重要课题。以联邦学习(Federated Learning) \cite{yang2019federated, jiang2019federated} 为代表的技术可以在保护各个研发机构数据隐私的前提下,提供一种挖掘跨机构数据背后的价值的计算范式。 通过应用联邦学习这种计算范式,整个行业可以有效解决数据孤岛问题,打造更强大的自然语言处理模型,实现一个互利共赢的业界生态。
\end{itemize}


综上所述,自然语言处理技术的发展同时面临着机遇与挑战。我们在体验着近年来自然语言处理的飞速发展带来的技术红利的同时,也期待在可见的将来该领域能在上述三个方面取得可喜的进展,从而带来自然语言处理相关行业的又一次腾飞。

